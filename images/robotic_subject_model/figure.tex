\documentclass{standalone}
%\documentclass{article}%just to see how much space it has taken
\usepackage{tikz}
\usetikzlibrary{arrows.meta, bending, shapes.misc, shapes.geometric}

% define colors and other constants
\colorlet{line color}{black!50}
\colorlet{light blue}{blue!40}
\colorlet{light gray}{gray!30}
\def\line thickness{1pt}
\def\left{-1}
\def\right{+1}
\newcommand{\xJointMarginX}{.5}
\newcommand{\xJointMarginY}{.6}
\newcommand{\verticalMargin}{.5}

% define primitive shapes for the drawing
\tikzset{
% line is used to connect various shapes
line/.style={
    -, draw=line color, line width=\line thickness},
%
% arrow with text
pics/text arrow/.style n args={2}{code={
    \draw[latex-, light blue, shorten <=2pt] (0.3*#2, 0.3)
        -- ++ (0.4*#2, 0.4) -- ++(0.4*#2, 0) node[above,
        black, font=\LARGE]{#1};}},
%
% revolute joint rotating in +x direction
pics/x plus joint/.style n args={2}{code={
    \draw[black, line width=\line thickness, fill=light gray]
        (0, 0.4) -- (0.3, 0) -- (0, -0.4) -- (-0.3, 0)
        -- cycle;
    \draw[black, line width=\line thickness] (0, -0.4)
        -- (0, 0.4);
    \fill[red] (-30:0.2) -- (-150:0.2) -- (90:0.2)
        -- cycle;
    \pic {text arrow={#1}{#2}};}},
%
% revolute joint rotating in +x direction
pics/x minus joint/.style n args={2}{code={
    \pic[rotate=180] {x plus joint={#1}{#2}};}},
%
% revolute joint rotating in +y direction
pics/y plus joint/.style n args={2}{code={
    \pic[rotate=-90] {x plus joint={#1}{#2}};}},
%
% revolute joint rotating in -y direction
pics/y minus joint/.style n args={2}{code={
    \pic[rotate=180] {x plus joint={#1}{#2}};}},
%
% revolute joint rotating in +z direction
pics/z plus joint/.style n args={2}{code={
    \draw[black, line width=\line thickness, fill=light gray]
        (0,0) circle (0.4);
    \draw[black, line width=\line thickness, fill=white]
        (0,0) circle (0.2);
    \draw[line width=2pt, red, -{Triangle[bend, length=6pt,
        width=8pt]}] (270:0.3) arc (270:150:0.3);
    \pic {text arrow={#1}{#2}};}},
%
% revolute joint rotating in -z direction
pics/z minus joint/.style n args={2}{code={
    \pic[yscale=-1, rotate=180] {z plus joint={#1}{#2}};}},
%
% shape to represent hand
hand/.pic={
    \draw node[draw=line color, rectangle, minimum height=8mm,
               minimum width=8mm, line width=\line thickness,
               chamfered rectangle, chamfered rectangle
               corners={south west, south east}, chamfered
               rectangle xsep=2pt, below] {};},
%
% shape to represent foot
foot/.pic={
    \draw node[draw=line color, rectangle, minimum height=4mm,
               minimum width=12mm, line width=\line thickness,
               below] {};},
%
% shape to represent head
head/.pic={
    \draw node[draw=line color, rectangle, minimum height=16mm,
               minimum width=14mm,line width=\line thickness,
               above] (box) {};
    \draw[line] ([yshift=-2mm] box.north west) --
                ([yshift=-2mm] box.north east);
    \draw[line] ([yshift=+2mm] box.south west) --
                ([yshift=+2mm] box.south east);};
}


\begin{document}
\begin{tikzpicture}
	% from head to spine shoulder
    \path[line] (0, 0) pic {head} |-
            ++(-\xJointMarginX,-.2*\xJointMarginY) |-
            ++(\xJointMarginX,-\xJointMarginY)
            pic {x plus joint={1}{\right}} -|
            ++(\xJointMarginX,-\xJointMarginY) -|
            ++(-\xJointMarginX,-\verticalMargin)
            pic {z minus joint={2}{\right}} --
            ++(0,-2*\verticalMargin)
            pic {y plus joint={3}{\right}} --
            ++(0,-.8*\verticalMargin)
            coordinate(spine shoulder);

	% from spine shoulder to spine base
    \path[line] (spine shoulder) |-
            ++(-\xJointMarginX,-\verticalMargin) |-
            ++(\xJointMarginX,-\xJointMarginY)
            pic {x plus joint={4}{\right}} -|
            ++(\xJointMarginX,-\xJointMarginY) -|
            ++(-\xJointMarginX,-\xJointMarginY)
            pic {y plus joint={5}{\right}} --
            ++(0,-2*\verticalMargin)
            pic {z plus joint={6}{\right}} |-
            ++ (-\xJointMarginX,-\xJointMarginY) |-
            ++(\xJointMarginX,-\xJointMarginY)
            pic {x plus joint={7}{\right}} -|
            ++(\xJointMarginX,-\xJointMarginY) -|
            ++(-\xJointMarginX,-\xJointMarginY)
            pic {y plus joint={8}{\right}} --
            ++(0,-\verticalMargin)
            coordinate(spine base);

	% from spine base to left foot
    \path[line] (spine base) -|
            ++(-3*\xJointMarginX,\xJointMarginY) -|
            ++(-\xJointMarginX,-\xJointMarginY)
            pic {z minus joint={21}{\left}} --
            ++(0,-\verticalMargin) |-
            ++(-\xJointMarginX,-\xJointMarginY) |-
            ++(\xJointMarginX,-\xJointMarginY)
            pic {x plus joint={20}{\right}} -|
            ++(\xJointMarginX,-\xJointMarginY) -|
            ++(-\xJointMarginX,-3*\xJointMarginY) -|
            ++(-\xJointMarginX,-\xJointMarginY) --
            ++(\xJointMarginX,0)
            pic {x plus joint={22}{\left}} -|
            ++(\xJointMarginX,-\xJointMarginY) -|
            ++(-\xJointMarginX,-3*\xJointMarginY)
            pic {foot};

	% right foot is mirror image of left foot
	\begin{scope}[xscale=-1, yscale=1]
    \path[line] (spine base) -|
            ++(-3*\xJointMarginY,\xJointMarginY) -|
            ++(-\xJointMarginX,-\xJointMarginY)
            pic {z minus joint={18}{\right}} --
            ++(0,-\verticalMargin) |-
            ++(-\xJointMarginX,-\xJointMarginY) |-
            ++(\xJointMarginX,-\xJointMarginY)
            pic {x plus joint={17}{\left}} -|
            ++(\xJointMarginX,-\xJointMarginY) -|
            ++(-\xJointMarginX,-3*\xJointMarginY) -|
            ++(-\xJointMarginX,-\xJointMarginY) --
            ++(\xJointMarginX,0)
            pic {x plus joint={19}{\right}} -|
            ++(\xJointMarginX,-\xJointMarginY) -|
            ++(-\xJointMarginY,-3*\xJointMarginY)
            pic {foot};
	\end{scope}

	% from spine shoulder to left hand
	\path[line] (spine shoulder) |-
	        ++(-5*\xJointMarginX,0)
            pic {x plus joint={14}{\left}} --
	        ++(-3*\xJointMarginX,0)
            pic {z plus joint={13}{\left}} --
	        ++(0,-2*\xJointMarginY)
            pic{y plus joint={15}{\left}} |-
            ++(-\xJointMarginX,-3*\xJointMarginY) |-
	        ++(\xJointMarginX,-\xJointMarginY)
            pic {x plus joint={16}{\left}} -|
            ++(\xJointMarginX,-\xJointMarginY) -|
	        ++(-\xJointMarginX,-3*\xJointMarginY)
            pic{hand};

	% right hand is mirror image of left hand
	\begin{scope}[xscale=-1, yscale=1]
	\path[line] (spine shoulder) |-
	        ++(-5*\xJointMarginX,0)
            pic {x plus joint={10}{\right}} --
	        ++(-3*\xJointMarginX,0)
            pic {z plus joint={9}{\right}} --
	        ++(0,-2*\xJointMarginY)
            pic{y plus joint={11}{\right}} |-
            ++(-\xJointMarginX,-3*\xJointMarginY) |-
	        ++(\xJointMarginX,-\xJointMarginY)
            pic {x plus joint={12}{\right}} -|
            ++(\xJointMarginX,-\xJointMarginY) -|
	        ++(-\xJointMarginX,-3*\xJointMarginY)
            pic{hand};
	\end{scope}
\end{tikzpicture}
\end{document}
